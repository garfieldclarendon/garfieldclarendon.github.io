A project that uses E\+S\+P8266 Wi\+Fi modules that communicate with a Raspberry Pi server to control accessories on the Garfield-\/\+Clarendon Model Railroad Club.

\subsubsection*{Prerequisites}

The code running on the E\+S\+P8266 is developed using the Arduino I\+DE

The L\+C\+S\+Server application running on the Raspberry Pi makes use of the Qt framework. The E\+S\+P8266 uses http calls to communicate with the L\+C\+S\+Server application and U\+DP unicast messages to communicate with other E\+S\+P8266 controllers on the layout. A separate R\+E\+S\+Tful A\+PI provides access to commands to activate routes, set turnouts and update data in the configuration database.

\href{https://garfieldclarendon.github.io/html/index.html}{\tt Source Code Documentation}

\href{https://garfieldclarendon.github.io/apidoc/index.html}{\tt A\+PI Documentation}

\subsection*{Built With}


\begin{DoxyItemize}
\item \href{https://www.arduino.cc/en/Main/Software}{\tt Arduino} -\/ Arduino Tools
\item \href{https://www.qt.io/}{\tt Qt} -\/ Qt Framework
\item \href{https://www.autodesk.com/products/eagle/overview}{\tt Eagle} -\/ Eagle Printed Circuit Board (P\+CB) design software
\end{DoxyItemize}

\subsection*{Contributing}

Please read \href{https://gist.github.com/PurpleBooth/b24679402957c63ec426}{\tt C\+O\+N\+T\+R\+I\+B\+U\+T\+I\+N\+G.\+md} for details on our code of conduct, and the process for submitting pull requests to us.

\subsection*{Authors}


\begin{DoxyItemize}
\item {\bfseries John Reilly} -\/ {\itshape Initial work}
\item {\bfseries Ryan Balla} -\/ {\itshape Web Development}
\end{DoxyItemize}

\subsection*{License}

This project is licensed under the M\+IT License -\/ see the L\+I\+C\+E\+N\+SE.md file for details 